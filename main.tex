%-----------------------------------------------------------------------------------------------------------------------------------------------%
%	The MIT License (MIT)
%
%	Copyright (c) 2019 Jan Küster
%
%	Permission is hereby granted, free of charge, to any person obtaining a copy
%	of this software and associated documentation files (the "Software"), to deal
%	in the Software without restriction, including without limitation the rights
%	to use, copy, modify, merge, publish, distribute, sublicense, and/or sell
%	copies of the Software, and to permit persons to whom the Software is
%	furnished to do so, subject to the following conditions:
%
%	THE SOFTWARE IS PROVIDED "AS IS", WITHOUT WARRANTY OF ANY KIND, EXPRESS OR
%	IMPLIED, INCLUDING BUT NOT LIMITED TO THE WARRANTIES OF MERCHANTABILITY,
%	FITNESS FOR A PARTICULAR PURPOSE AND NONINFRINGEMENT. IN NO EVENT SHALL THE
%	AUTHORS OR COPYRIGHT HOLDERS BE LIABLE FOR ANY CLAIM, DAMAGES OR OTHER
%	LIABILITY, WHETHER IN AN ACTION OF CONTRACT, TORT OR OTHERWISE, ARISING FROM,
%	OUT OF OR IN CONNECTION WITH THE SOFTWARE OR THE USE OR OTHER DEALINGS IN
%	THE SOFTWARE.
%
%
%-----------------------------------------------------------------------------------------------------------------------------------------------%

\input{setup}

\newcommand{\emailAddress}{Korbad@gmail.com}

%============================================================================%
%
%
%
%	DOCUMENT CONTENT
%
%
%
%============================================================================%
\begin{document}
\columnratio{0.31}
\setlength{\columnsep}{2.2em}
\setlength{\columnseprule}{4pt}
\colseprulecolor{lightcol}
\begin{paracol}{2}
\begin{leftcolumn}
%---------------------------------------------------------------------------------------
%	META IMAGE
%----------------------------------------------------------------------------------------
%\includegraphics[width=\linewidth]{untitled.jpg}	%trimming relative to image size


\vfill\null
\cvsection{CONTACT}

\iconemail{EnvelopeSquare}{14}{Korbad@gmail.com}{Korbad@gmail.com}{black}\\[6pt]
\icontext{Github}{14}{\href{https://github.com/Korbad}{Korbad}}{black}\\[6pt]
\icontext{Linkedin}{14}{\href{https://www.linkedin.com/in/joshpperry/}{joshpperry}}{black}\\[6pt]
\vfill\null
%\cvqrcode{0.7}

%---------------------------------------------------------------------------------------
%	META SKILLS
%----------------------------------------------------------------------------------------
\cvsection{SKILLS}

%\cvskill{Skill_Name} {Years of experience} {percentage of bar fill} \\[-2pt]

\cvskill{Python} {5+ yrs} {0.8} \\[-2pt]

\cvskill{Physics} {10+ yrs} {1} \\[-2pt]

\cvskill{Mathematics} {3+ yrs} {0.6} \\[-2pt]

\cvskill{C++} {2+ yrs} {0.2} \\[-2pt]

\cvskill{Linux} {10+ yrs} {0.8} \\[-2pt]

\cvskill{Quantum Mechanics} {10+ yrs} {0.8} \\[-2pt]

\cvskill{Statistics} {5+ yrs} {0.8} \\[-2pt]


%\vfill\null
%\cvqrcode{0.7}

%---------------------------------------------------------------------------------------
%	ACHIEVEMENTS
%----------------------------------------------------------------------------------------
\newpage
\cvsection{ACHIEVEMENTS}

\cvmetaevent
{Internship for PENIX Spinfest 2011}
{RIKEN Institute, Chiba, Japan}
{}
{Awarded internship}

\end{leftcolumn}
\begin{rightcolumn}
%---------------------------------------------------------------------------------------
%	TITLE  HEADER
%----------------------------------------------------------------------------------------
\fcolorbox{white}{darkcol}{\begin{minipage}[c][3.5cm][c]{1\mpwidth}
	\begin {center}
		\HUGE{ \textbf{ \textcolor{white}{ \uppercase{ JOSHUA PERRY } } } } \\[-24pt]
		\textcolor{white}{ \rule{0.1\textwidth}{1.25pt} } \\[4pt]
		\large{ \textcolor{white} {Research Scholar - Nuclear Physics} }
	\end {center}
\end{minipage}} \\[14pt]
\vspace{-12pt}

%---------------------------------------------------------------------------------------
%	EDUCATION
%----------------------------------------------------------------------------------------
%\vfill\null
\cvsection{EDUCATION}

\cvevent
	{\textbf{2008 - 2014}}
	{Ph. D. - Nuclear Physics}
	{Iowa State University - Ames, IA (USA)}
	{Ph. D. awarded}
\vfill\null

\cvevent
	{\textbf{2005 - 2008}}
	{Bachelor of Science - Mathematics}
	{Winona State University - Winona, MN (USA)}
	{Gradated}
\vfill\null

\cvevent
	{\textbf{2005 - 2008}}
	{Bachelor of Science - Physics}
	{Winona State University - Winona, MN (USA)}
	{Gradated}
\vfill\null

%---------------------------------------------------------------------------------------
%	WORK EXPERIENCE
%----------------------------------------------------------------------------------------
\vfill\null
\cvsection{WORK EXPERIENCE}

\cvevent
	{\textbf{Jan 2016 - Present}}
	{Automation Engineer}
	{IBM, Remote (work-from-home for entire duration)}
	{Work with stakeholders to find and automate inefficient processes}
	{}
\vfill\null

\cvevent
	{\textbf{May 2015 - Jan 2016}}
	{Infrastructure and Endpoint Security Specialist}
	{IBM, Dubuque, IA (USA)}
	{Create novel solutions to automate manual processes.}
	{}
\vfill\null
%---------------------------------------------------------------------------------------
%	PUBLICATION
%----------------------------------------------------------------------------------------
\vspace{-0.5cm}
\vfill\null
\cvsection{PUBLICATIONS}

\cvevent
	{\textbf{Ph. D. Thesis}}
	{A Novel Approach to Central-Jet Forward-Hadron Correlations in Proton-Proton Collisions at 200 GeV at the PHENIX Experiment}
	{Publisher: Iowa State University (2015)}
	{Status: Accepted and Published}
	{The results of Semi-Inclusive Deep Inelastic Scattering and Deep Inelastic Scattering experiments combined in a global analysis have shown us that a transversely polarized quark undergoes an azimuthal spatially-biased neutral pion fragmentation. One goal of the PHENIX Experiment at RHIC is to study similar properties in a hadron-hadron collision environment in an attempt to show the degree to which universality is broken between these differing systems as well as the relative contribution of competing processes to the observed single spin asymmetries. A novel measurement method was formulated specifically for the PHENIX experiment's detector configuration. The method and results are presented here.}
\vfill\null


%---------------------------------------------------------------------------------------
%	PROJECTS
%----------------------------------------------------------------------------------------
\vfill\null
\cvsection{PROJECTS}

\cvevent
	{\textbf{20XX}}
	{Muon-Piston Calorimeter Upgarde (MPC-EX)}
	{Pre-shower detector placed in front of the existing calorimeter}
	{}
\vfill\null


%---------------------------------------------------------------------------------------
%	WORKSHOPS
%----------------------------------------------------------------------------------------
\vfill\null
\cvsection{WORKSHOPS \& CONFERENCES}

\cvevent
	{\textbf{September 17-22, 2012}}
	{International Spin Physics Symposium}
	{Joint Institute for Nuclear Research, Dubna, Russia}
	{}
\vfill\null

%---------------------------------------------------------------------------------------
%	SKILLS
%----------------------------------------------------------------------------------------

%---------------------------------------------------------------------------------------
%	PERSONAL DETAILS
%----------------------------------------------------------------------------------------
% \vfill\null
% \cvsection{EXTRACURRICULAR}
% \vspace{-0.3cm}
% \begin{itemize}
%   \item Put all the points that are not covered in \textbf{above sections}.
%   \item Put all the points that are \textbf{not covered} in above sections.
%   \item Put all the \textbf{points} that are not covered in above sections.
%   \item \textbf{Put all the points} that are not covered in above sections.
% \end{itemize}
% \vfill\null


% hotfixes to create fake-space to ensure the whole height is used
\vfill
\vfill
\vfill
\end{rightcolumn}
\end{paracol}
\end{document}
